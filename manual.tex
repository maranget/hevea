\documentclass[a4paper]{article}
\usepackage{isolatin1}

\newcommand{\htmlgen}{{\sf h\raise .2ex\hbox{T}mlg\lower .1ex\hbox{e}n}}
\begin{document}

\section{Introduction}
The programm \htmlgen\ is a \LaTeX\ to html translator.
The input language is an almost complete subset of latex2e and the
output langaguage is html that is (hopefullly) correct with respect to
version 3.2.

Mathematics and other exotic symbols are translated into symbols
pertening to the symbol font of the html browser, using the
non-standard \verb+FACE+ atribute of the \verb+FONT+ tag.
This allows the translation to html of most of the symbols used in
\LaTeX. If necessary this behavior can altered and \htmlgen\ will
produde english of french text equivalents of these symboles (For
instance $\in$ will be translated into {\em in} or {\em appartient
\`a}).


\htmlgen\ understands user \LaTeX macros, given both in \LaTeX\ style
(using \verb+\newcommand+ and \verb+\renewcommand+) and in sensible
\TeX\ style (using \verb+\def#1#2...+ without delimiting
characters). This allows to use user style
files with little or no modifications, provided these style files are
not written in \TeX\ and are not too tricky.

\htmlgen is written in Objective Caml, as many lexers, it is
quite fast and flexible.
Using \htmlgen\ is is possible to translate  big documents such
as manuals, books, etc. very quickly. All documents are
translated as one single html file. Then, the output file can be cut into
smaller files (one file per chapter) using the companion program \htmlcut.



\section{How to get started}

Assume that you have a file \verb+a.tex+ written in \LaTeX, using the
\verb+article+ style. Then, translation is achieved by issuing the
command~:
\begin{verbatim}
     htmlgen article.sty a.tex
\end{verbatim}

If everything goes fine, this will produce a new file
``\verb+a.html+'' that you can visualize using an html brower.
If \verb+a.tex+ contains maths symbols you need to instruct your
browser to use symbol fonts. For Netscape, add the following line to
your \verb+.Xdefaults+ file 
\begin{verbatim}
Netscape*documentFonts.charset*adobe-fontspecific:   iso-8859-1
\end{verbatim}
(Then, you probably need something like a \verb+xrdb+ command before
you start Netscape.)


If you wish to experiment \htmlgen\ on small \LaTeX\ source fragments,
then lauch \htmlgen without arguments. \htmlgen\ will read its
standard input and print the translation on its standard output.
For instance:
\begin{verbatim}
% htmlgen
$$ 
x \in {\cal E}
$$
% (this latex comment is here for flushing pending output)
<DIV ALIGN=center>
<TABLE VALIGN=middle CELLSPACING=0 CELLPADDING=0>
<TR><TD nowrap><I> x <FONT FACE=symbol>�</FONT> </I></TD>
<TD nowrap><I><FONT COLOR=red> E</FONT></I></TD>
</TR>
</TABLE>
</DIV>
\end{verbatim}



\end{document}