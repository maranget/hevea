\documentclass{article}
\usepackage[latin1]{inputenc}
\usepackage{moreverb}
\begin{document}

\section*{Les environnements de positionnement}
\subsection*{D'abord quelques imbriqu�s simples}
\begin{center}
Je centre.
\begin{flushleft}
Je fer � gauche.
\begin{flushright}
Je fer � droite.
\end{flushright}
\end{flushleft}
\end{center}

\subsection*{Ensuite quelques tables dans les flush}
Il faut positionner la table, mais ne pas h�riter des
positionnements de texte (si j'ai bien compris).
\begin{center}
Je centre une table.\\
\begin{tabular}{rcl}
Coucou & = & Z\\
Z & = & Coucou (zobi, la mouche)\\
\end{tabular}
\begin{flushleft}
Je fer � gauche.\\
\begin{tabular}{rcl}
Coucou & = & Z\\
Z & = & Coucou (zobi, la mouche)\\
\end{tabular}
\begin{flushright}
Je fer � droite.\\
\begin{tabular}{rcl}
Coucou & = & Z\\
Z & = & Coucou (zobi, la mouche)\\
\end{tabular}
\end{flushright}
\end{flushleft}
\end{center}


\section*{Les \texttt{verbatim} dans les environnements}
\begin{boxedverbatim}
Coucou j'essaie le verbatim dans une bo�te.
�a a l'air de fonctionner.
\end{boxedverbatim}

\begin{flushright}
\begin{verbatim}
Coucou j'essaie le verbatim dans un flushright.
Et bien �a a l'air de fonctionner.
C'est � dire que c'est bien � gauche !
\end{verbatim}
\end{flushright}

\begin{center}
\begin{verbatim}
Coucou j'essaie le verbatim dans un center.
Et bien �a a l'air de fonctionner.
C'est � dire que c'est bien � gauche !
\end{verbatim}
\end{center}

\end{document}
