\documentclass{article}
\usepackage{bussproofs}
\usepackage{amssymb}
\newtheorem{definition}{Definition}
\newtheorem{lemma}{Lemma}
\newtheorem{theorem}{Theorem}

\begin{document}
\title{Simply Typed Lambda Calculus}
\maketitle
The simply typed $\lambda$-calculus is given by the rules\\ \\
$
\AxiomC{}
\LeftLabel{(Ax)}
\RightLabel{where $x : p \in \Gamma$}
\UnaryInfC{$\Gamma \vdash x : p$}
\DisplayProof
$

$
\AxiomC{$\Gamma \vdash t : \bot$}
\LeftLabel{($\neg$ E)}
\RightLabel{where $p \not = \bot$}
\UnaryInfC{$\Gamma \vdash \bigtriangledown_{p} (t) : p$}
\DisplayProof
$
                                                                                
$
\AxiomC{$\Gamma, x : p_1 \vdash t : p_2$}
\LeftLabel{($\to$ I)}
\UnaryInfC{$\Gamma \vdash \lambda (x:p_1).t : p_1 \to p_2$}
\DisplayProof
$

$
\AxiomC{$\Gamma \vdash t_1 : p_1 \to p_2$}
\AxiomC{$\Gamma \vdash t_2 : p_1$}
\LeftLabel{($\to$ E)}
\BinaryInfC{$\Gamma \vdash t_1 t_2 : p_2 $}
\DisplayProof
$

$
\AxiomC{$\Gamma \vdash t_1 : p_1$}
\AxiomC{$\Gamma \vdash t_2 : p_2$} 
\LeftLabel{($\wedge$ I)}
\BinaryInfC{$\Gamma \vdash \langle t_1 t_2 \rangle : p_1 \wedge p_2 $}
\DisplayProof
$

$
\AxiomC{$\Gamma \vdash t : p_1 \wedge p_2$} 
\LeftLabel{($\wedge$ E$_1$)}
\UnaryInfC{$\Gamma \vdash proj_1 t : p_1$}
\DisplayProof
$

$
\AxiomC{$\Gamma \vdash t : p_1 \wedge p_2$} 
\LeftLabel{($\wedge$ E$_2$)}
\UnaryInfC{$\Gamma \vdash proj_2 t : p_2$}
\DisplayProof
$

$
\AxiomC{$\Gamma \vdash t : p_1$} 
\LeftLabel{($\vee$ I$_1$)}
\UnaryInfC{$\Gamma \vdash in_1 t : p_1 \vee p_2 $}
\DisplayProof
$

$
\AxiomC{$\Gamma \vdash t : p_2$} 
\LeftLabel{($\vee$ I$_2$)}
\UnaryInfC{$\Gamma \vdash in_2 t : p_1 \vee p_2 $}
\DisplayProof
$

$
\AxiomC{$\Gamma \vdash t : p_1 \vee p_2$}
\AxiomC{$\Gamma, x_1 : p_1 \vdash u_1 : q$}
\AxiomC{$\Gamma, x_2 : p_2 \vdash u_2 : q$}
\LeftLabel{($\vee$ E)}
\TrinaryInfC{$\Gamma \vdash case (t, \lambda (x_1:p_1).u_1, \lambda (x_2:p_2).u_2) : q$}
\DisplayProof
$

In the next four rules, $i$ stands for the type of propositions\\ \\
$
\AxiomC{$\Gamma \vdash t : p[\tau/P]$}
\LeftLabel{($\forall$ I)}
\RightLabel{where$~\tau\not\in fv(\Gamma\cup\{\forall~P~p\})$}
\UnaryInfC{$\Gamma \vdash \lambda (\tau:i).t : \forall P~p$}
\DisplayProof
$

$
\AxiomC{$\Gamma \vdash t : \forall P ~p$}
\LeftLabel{($\forall$ E)}
\UnaryInfC{$\Gamma \vdash t \tau : p[\tau/P]$}
\DisplayProof
$

$
\AxiomC{$\Gamma \vdash t : p[\tau/P]$}
\LeftLabel{($\exists$ I)}
\UnaryInfC{$\Gamma \vdash inx (\tau, t) : \exists P p$}
\DisplayProof
$

$
\AxiomC{$\Gamma \vdash t : \exists P ~p$}
\AxiomC{$\Gamma, x : p[\tau/P] \vdash u$}
\LeftLabel{($\exists$ E)}
\RightLabel{where$~\tau\not\in fv(\Gamma\cup\{\forall~P~p\}\cup{q})$}
\BinaryInfC{$\Gamma \vdash casex (t, \lambda (\tau:i).\lambda (x:p[\tau/P]).u) : q$}
\DisplayProof
$

\end{document}


