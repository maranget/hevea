\documentclass{article}

\usepackage{alltt}
\usepackage{html}
\newcommand{\rbox}[2]{$^{\mbox{#2}}$}
\newcommand{\lbox}[2]{$_{\mbox{#2}}$}
\begin{latexonly}
\gdef\myrule{\rule{10cm}{.05ex}\\[-.5ex]}
\end{latexonly}
\begin{htmlonly}
\gdef\myrule{\@print{<HR NOSHADE SIZE=1 ALIGN=center WIDTH=75%>
}}%
\gdef\suite{\@print{<HR NOSHADE SIZE=1 ALIGN=center WIDTH=25%>
}}%
\end{htmlonly}
\newenvironment{htmlout}{\begin{quote}\center\myrule}{~\\[-1ex]\myrule\endcenter\end{quote}}
\title{How to spot and correct some trouble}
\begin{document}
\maketitle

\section{\htmlgen\  ignores a macro}

\subsection{Trouble}
Consider the following \LaTeX\ source excerpt:
\begin{verbatim}
\begin{quote}
You can \raisebox{.6ex}{\em raise} or \raisebox{-.6ex}{\em lower}
text.
\end{quote}
\end{verbatim}
Since \htmlgen\ does not know about \verb+raisebox+, it will
incorrectly process this input. More precisely,
it first prints a warning message:
\begin{verbatim}
trouble.tex:34: Unknown macro: \raisebox
\end{verbatim}
Then, it goes on reading the arguments of \verb+\raisebox+ as normal text. As a
consequence some \verb+.6ex+ is finally found in the html output:
\begin{htmlout}
\begin{latexonly}
You can .6ex{\em raise} or -.6ex{\em lower text}.
\rule[.1ex]{5cm}{.4ex}\\
You can \raisebox{.6ex}{\em raise} or \raisebox{-.6ex}{\em lower}
text.
\end{latexonly}
\begin{htmlonly}
You can \raisebox{.6ex}{\em raise} or \raisebox{-.6ex}{\em lower}
text.
\suite
You can \rbox{.6ex}{\em raise} or \lbox{-.6ex}{\em lower}
text.
\end{htmlonly}
\end{htmlout}


\subsection{Patch}

A first patch is to forget altogether about raising or lowering text,
by defining \verb+\raisebox+ as a two arguments macro that ignores its
first argument.
\begin{verbatim}
\newcommand{\raisebox}[2]{{#2}}
\end{verbatim}
The best place for this definition is an user-style file (say
\verb+trouble.sty+) given as
argument to \htmlgen.
Then, issue the follwing command:
\begin{verbatim}
# htmlgen trouble.sty trouble.tex
\end{verbatim}
Both \LaTeX\ and \htmlgen\ correctly process the example. However
there is no more raising or lowering in the produced hml.
\begin{htmlout}\renewcommand{\raisebox}[2]{{#2}}
You can \raisebox{.6ex}{\em raise} or \raisebox{-.6ex}{\em lower}
text.
\end{htmlout}

If you stick to raising or lowering text, then you need change the
text itself. Write two macros:
\begin{verbatim}
\newcommand{\raisebox}[2]{$^{\mbox{#2}}$}
\newcommand{\lowerbox}[2]{$_{\mbox{#2}}$}
\end{verbatim}
And then, use \verb+\lowertext+ where appropriate.
\begin{verbatim}
\begin{htmlout}
You can \raisebox{.6ex}{\em raise} or
\lowerbox{-.6ex}{\em lower} text a little.
\end{quote}
\end{verbatim}
We get:
\begin{htmlout}
\renewcommand{\raisebox}[2]{$^{\mbox{#2}}$}%
\newcommand{\lowerbox}[2]{$_{\mbox{#2}}$}
You can \raisebox{.6ex}{\em raise} or \lowerbox{-.6ex}{\em lower}
text a little.
\end{htmlout}


\subsection{Advanced patch}

\begin{htmlonly}
\gdef\raisebox#1#2{\@open{}{}\@print{<SUP>}#2\@print{</SUP>}\@close{}}
\gdef\lowerbox#1#2{\@open{}{}\@print{<SUB>}#2\@print{</SUB>}\@close{}}
\end{htmlonly}
\begin{latexonly}
\gdef\lowerbox#1#2{\raisebox{#1}{#2}}
\end{latexonly}
\begin{htmlout}
You can \raisebox{.6ex}{\em raise} or \lowerbox{-.6ex}{\em lower}
text a little.
\end{htmlout}
\section{\htmlgen\  uncorrectly interprets a macro}

\subsection{Trouble}
Sometime, \htmlgen\  

This will not work. \htmlgen\  translate the \LaTeX macro
\verb+\rule+. Thus, it prints a wraning message and then goes on
processing the arguments as normal text.

\begin{htmlout}\newcommand{\blob}{\rule[.2ex]{1ex}{1ex}}
\blob\ Blob \blob
\end{htmlout}

\section{An alternative}

Now a blob is a diamond:
\begin{htmlout}\newcommand{\blob}{$\diamondsuit$}
\blob\ Blob \blob
\end{htmlout}


\section{\htmlgen\ crashes with a \protect\texttt{html:} failure}




\end{document}