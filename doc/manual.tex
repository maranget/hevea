\documentclass[a4paper]{article}
\usepackage{../html}
\usepackage{isolatin1}
\usepackage{a4wide}

\newcommand{\urlex}[1]{\url{\docurl/examples/#1}{\filename{#1}}}
\newcommand{\command}[1]{{\tt #1}}
\newcommand{\filename}[1]{{\tt #1}}

\title{~\htmlgen\ user documentation}
\author{Luc Maranget\thanks{Inria Rocquencourt -- BP 105, 78153 Le
Chesnay Cedex. {\tt Luc.Maranget@inria.fr}}}

\begin{document}
\cutdef{document}{part}
\maketitle

\begin{abstract}
The programm \htmlgen\ is a \LaTeX\ to html translator.
The input language is a fairly complete subset of \LaTeX2e\ (old
\LaTeX\ style is also accepted) and the
output langaguage is html that is (hopefullly) correct with respect to
version 3.2.

Mathematics and other exotic symbols are translated into symbols
pertening to the symbol font of the html browser, using the
non-standard \verb+FACE+ atribute of the \verb+FONT+ tag.
This allows the translation to html of quite a lot of the symbols used in
\LaTeX.


\indent\htmlgen\ understands \LaTeX\ user macros. This allows to use user style
files with little or no modifications, provided these style files are
not too tricky.

\indent\htmlgen\ is written in Objective Caml, as many lexers. It is
quite fast and flexible.
Using \htmlgen\ is is possible to translate  big documents such
as manuals, books, etc. very quickly. All documents are
translated as one single html file. Then, the output file can be cut into
smaller file, using the companion program \htmlcut.
\end{abstract}


\newpage

\part{Tutorial}

\section{How to get started}\label{getstarted}

Assume that you have a file \filename{a.tex} written in \LaTeX, using the
\filename{article}, \filename{book} or \filename{report} style. Then,
translation
is achieved by issuing the command~:
\begin{verbatim}
     htmlgen a.tex
\end{verbatim}
Probably, you will get some warnings about unknown macros. If
\htmlgen\ does not crash, just ignore them for the moment.

\label{browser}If everything goes fine, this will produce a new file
``\filename{a.html}'' that you can visualize using an html brower.
If \filename{a.tex} contains maths symbols you need to instruct your
browser to use symbol fonts. For Netscape, add the following line to
your \filename{.Xdefaults} file 
\begin{verbatim}
Netscape*documentFonts.charset*adobe-fontspecific:   iso-8859-1
\end{verbatim}
(Then, you probably need something like a \command{xrdb} command before
you start Netscape.)


If you wish to experiment \htmlgen\ on small \LaTeX\ source fragments,
then lauch \htmlgen\ without arguments. \htmlgen\ will read its
standard input and print the translation on its standard output.
For instance:
\begin{verbatim}
# htmlgen
$
x \in {\cal E}
$
^D
<I>
x <FONT FACE=symbol>�</FONT> </I><I><FONT COLOR=red> E</FONT></I><I>
</I>
\end{verbatim}

You can find some \url{\docurl/examples/}{examples} in the on-line
documentation, including examples of \urlex{a.tex} and
\urlex{a.html} files.

\section{More ambitious translations}

\subsection{Base style files}

\noindent\htmlgen\ knows about three \LaTeX\ base style file, \verb+article+,
\verb+book+ and \verb+report+. In case your source file uses
another base style, then you can instruct \htmlgen to load one of the
style files it knows about before processing your source file.
This is done by giving the desired style file as an extra command line
argument.

Let us assume that \verb+acmarticle.tex+ uses an exotic style
file such as \verb+acmconf.sty+. Them, typing
\verb+htmlgen acmarticle.tex+ may yield two kind of errors.
Either \htmlgen\ finds the the \verb+acmconf.sty+ file (e.g. if you have a
private copy of it in your directory) and fails to
interpret it~:
\begin{verbatim}
# htmlgen acmarticle.tex
./acmconf.sty:84: Unknown macro: \typeout
./acmconf.sty:85: Unknown macro: \typeout
./acmconf.sty:87: Adios
Fatal error: uncaught exception Failure("Bad newif: \if")
\end{verbatim}
Or,  \htmlgen\ cannot find the \verb+acmconf.sty+ file and you get a
lot of warnings:
\begin{verbatim}
# htmlgen acmarticle.tex
main.tex:9: Unknown macro: \@docpart
main.tex:9: Unknown macro: \@docchapter
main.tex:18: Unknown counter: footnote
...
\end{verbatim}

Both situations are avoided by issuing the command
\verb+htmlgen article.sty acmarticle.sty+ that instruct
\htmlgen\ to load its \verb+article.sty+
style file in place of the document style file.

\subsection{Macros}
Just like \LaTeX, \htmlgen\ can be seen as a macro langaguage, macros
are rewritten until no more expansion is possible. Then either some
caracters (such as letters, integers\ldots) are outputed or some
internal operation (such as changing font attributes, or arranging
text items in a certain manner) are performed.

\subsection{Base macros}
\LaTeX\ and html differ in many aspects. For instance, \LaTeX\ allows
much flexibility in spacing, raising or lowering texts items, whereas
html does not.
Much more symbols and font attributes are available in \LaTeX\ than in
html. However, html has font attributes, such as color, which \LaTeX\
has not.

As a consequence, there are many situation where \htmlgen\ just cannot
render the visual effect of \LaTeX\ constructions. Here some choices
have to be made. For instance, the calligraphic letters (\verb+\cal+)
are rendered in red (\verb+<FONT COLOR=red>+), and the small caps
(\verb+\sc+) are rendered in bold font (\verb+<BF>+).

If you are not satisfied with my choice, then you
can make your own choice, by redefining the \verb+\cal+ and \verb+\sc+
macros. To avoid modyfing your \LaTeX\ file, it is advised to use an
extra-style file that contains your definitions. Thus in a
\verb+mydefs.sty+ file you can type:
\begin{verbatim}
\renewcommand{\sc}{\large}
\renewcommand{\cal}{\em}
\end{verbatim}
Then, you process a tex file \verb+testsccal.tex+ by the command:
\begin{verbatim}
htmlgen mydefs.sty testsccal.tex
\end{verbatim}
You find the files
\urlex{testsccal.tex}
\urlex{testsccal.sty}
and the result file \urlex{testsccal.html} in the on-line documentation.


Note that base macros are partly defined in \LaTeX\ style in the
\verb+htmlgen.sty+ file that \htmlgen\ loads before processing any
input. Other base macros that require a special processing are defined
in \htmlgen source code.
Having a look at the \verb+htmlgen.sty+ file (or at the base style
files of \htmlgen) will help you in
designing your own macros.

Hopefully, \htmlgen understands most of \LaTeX base macros and environments.
This includes positionning environemnts such as \verb+center+,
\verb+flushleft+ or \verb+quote+; sectioning commands, such as
\verb+\section+; most of font atribute
changes such as \verb+\em+, \verb+\tt+, \verb+\large+ and the
associated environements; math mode as introduced by \verb+$+ and
display mode as introduced by \verb+$$+\ldots; subscripting and
superscriting; many math symbols, such as \verb+\in+ ($\in$) or
\verb+\leftarrow+ ($\leftarrow$); array and tabular
environmnents; cross references, figures, indexes, footnotes and
bibliographic references are
also handled. Again, I made some choices while mapping \LaTeX\
constructs onto html and some part of them can be modified by
rewriting macros and environments in \LaTeX.

The html implementation of some nasty \TeX\ or \LaTeX\ constructs such
as \verb+\over+, \verb+\left...\right+ or limits would deserve a more
thorough description. Some others such as the \verb+tabbing+ environment would
deserve a more thourough implementation.
Some may have deserved an implementation, such as \verb+\sqrt+\ldots\ 
I will not say more about them, you are invited to experiment and send
your remarks or critics to \mail{Luc.Maranget}{inria.fr}.
You can also have a look at \htmlgen\ test suite, see \url{suite.tex}
and \urlex{suite.html}.

\subsection{User macros}
\noindent\htmlgen\ understands user macros,
given both in \LaTeX\ style
(using \verb+\newcommand+, \verb+\renewcommand+ and
\verb+\providecommand+) and in simple
\TeX\ style (using \verb+\[g]def#1#2...+ without delimiting
characters).
It also processes the \verb+\[re]newenvironment+ constructs and a
limited version of \verb+\let+.


In good cases, when the macros does not perform subtle typesetting or
overuse \TeX\ constructs, everything goes fine.
However it may happen that the visual effect obtained is not
satisfactory, even worse some spurious output such as arguments of
unknown macros may get their way into the final html output.

Consider another example, \urlex{trouble.tex}, which includes the
following macro definition and usage:
\begin{verbatim}
\newcommand\blob{\hrule height 1ex length 1ex}
\end{verbatim}


\section{Other style files}

A \LaTeX\ document usually loads additional style files (using
\verb+\input+ or \verb+\usepackage+).
\htmlgen\ simply ignores \verb+\usepackage+ but attempt to load
input files.
Appart from understanding the warning of \htmlgen, you can see which
files it loads by using the \verb+-v+ option.

Consider again a simple \verb+article.tex+ article whose header is as follows

\part{Reference manual}

\section{Usage}
\begin{verbatim}
# htmlgen -help
htmlgen 0.0
  -v , verbose flag, can be repeated to increase verbosity
  -e filename, prevent file ``filename'' from being read
  -idx , attempt to read .idx file (useful if indexing is non-standard)
  -francais , french mode
  -nosymb , do not output symbol fonts
  -I dir, add directory ``dir'' to search path
  -o filename, make htmlgen output go into file ``filename''
\end{verbatim}

\begin{verbatim}
# htmlcut 0.0
  -o filename, make htmlcut output go into file ``filename'' (defaults to index.html)
\end{verbatim}

\part{Installation}

\section{Requirements}

This package is mainly written in
\url{http://caml.inria.fr/ocaml/}{Objective Caml}. Thus, you
really need Objective Caml to compile it.

The \htmlgen\ user may instruct the program not to process a
part of the input. Instead, this parts is processed into a
\verb+.gif+ file and \htmlgen\ ouptuts a  link to the image file.
\LaTeX\ is changed into a \verb+.gif+ file by the the \verb+imagegen+
script, which basically calls, \LaTeX, ghostscript a few tools from
the image processing package
\url{ftp://wuarchive.wustl.edu/graphics/graphics/packages/NetPBM}{netpbm}.
To benefit from the full functionnality of \htmlgen, you need all
this software. However, \htmlgen\ runs without them, but then you will
have to manage to produce images by yourself.

\subsection{Principles}
The details are given in the \verb+README+ file in the distribution.



\end{document}